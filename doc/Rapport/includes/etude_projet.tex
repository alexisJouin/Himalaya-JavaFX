\section{Étude du projet}
\subsection{Contraintes et situations initiales}
Le projet initial à réaliser est un bot en \java pour le jeu de société \himalaya. Le moteur du jeu a déjà été réalisé par l'Université de Nice. Cependant, pour des raisons techniques et pratiques nous avons dû repartir à zéro afin de construire un moteur de jeu plus adapté pour une interface graphique (avec \fx) et les intelligences artificielles. Nous nous sommes tout de même inspirés du travail réalisé précédemment.

\subsection{Pourquoi \fx ?}
\fx contient des outils et composants graphiques les plus maintenus du langage \java.
De plus, cette <<API>> est très rapide et simple à utiliser.
Cela nous a permis en plus d’apprendre et de maitriser un nouvel outil de développement d’interface graphique en \java.

\subsection{Objectifs à réaliser}
Les objectifs principaux du projet ont été de :
\begin{itemize} 
	\item Construire le moteur du jeu 
	\item Construire l'interface graphique
	\item Construire l’intelligence artificielle du jeu
\end{itemize}
