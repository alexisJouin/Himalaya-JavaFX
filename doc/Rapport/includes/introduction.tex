\section{Introduction}
Le projet de fin d'année de Master ISIDIS est très intéressant et important pour nous initier à un travail de recherche.
Nous avons fait le choix de travailler sur le projet \himalaya en binôme, ce qui implique tout de même une organisation dans les tâches de travaux.
On a fait ce choix, car on était intéressé de développer un bot en \java pour un jeu de société.
De plus, ce projet nous a permis d’utiliser de nouvelles technologies que nous ne maîtrisions pas comme \fx.
Notre binôme est composé Sébastien Schouteeten et Alexis Jouin.
À partir de ce constat, nous avons donc essayé de réaliser un logiciel fonctionnel, remplissant les conditions imposées par le cahier des charges validé par monsieur Fonlupt. Nous allons donc voir à travers ce rapport dans une première partie, une présentation du projet ainsi que ses principaux objectifs. Puis dans une seconde partie, les méthodes que nous avons utilisées afin de mettre en \oe uvre le projet et son élaboration. Enfin, dans la dernière partie, nous verrons les résultats obtenus ainsi que les évolutions possibles du projet et plus particulièrement du logiciel.