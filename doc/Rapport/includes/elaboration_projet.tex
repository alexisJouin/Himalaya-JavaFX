\section{Élaboration du projet}

\subsection{Les différentes phases du projet}
\subsubsection{Création du moteur du jeu}

Nous avons débuté le projet par la création du moteur du jeu.
Ce moteur a été pensé à l'avance avec la réalisation de l'\emph{UML},
représentant les classes \java.
Nous nous sommes également aidés du code source du jeu qui nous a été fourni
en début de projet.

Le moteur du jeu est réalisé de manière à pouvoir être utilisé aussi bien en
console (terminal), qu'avec une interface graphique.

Durant la réalisation du moteur, nous avons mis en place des tests unitaires,
dans le but de pouvoir tester notre code, et vérifier qu'aucune des
fonctionnalités précédentes n'est altérée par les nouvelles.

Nous avons réalisé une classe \emph{Player.java} qui représente un joueur humain,
et nous avons créé ses classes filles pour utiliser les intelligences artificielles.

\subsubsection{Création de l'interface}

Une fois le moteur terminé et les tests unitaires finis, nous avons réalisé une interface en console.
Cette dernière nous a permis de réaliser des tests avec une vraie personne qui joue, et de 
commencer à mettre en place une intelligence artificielle.

Par la suite, nous avons commencé à travailler sur une interface graphique
du jeu. Nous avons choisi d'utiliser \fx pour réaliser cette étape, car c'est
la méthode la plus rapide pour mettre en place une interface fonctionnelle en \java.
En effet, \fx est maintenant la bibliothèque de création d'interfaces graphiques officielle du langage \java.

Nous avons réalisé un menu \reference{menu}, qui permet de sélectionner les joueurs et leur type (Humain, IA aléatoire ou Évolutionnaire). Il est possible, depuis ce menu, de passer en mode 3 ou 4 joueurs.

Juste avant de pouvoir jouer, le jeu demande aux joueurs de choisir leur position de départ \reference{pos_init}.

Durant le déroulement du jeu, nous affichons le plateau ainsi que les ressources
possédées pour chaque joueur, ainsi que les commandes et ressources proposées sur
les villages. Les pions des joueurs ainsi que les délégations et stupas qu'ils ont posés
apparaissent également sur le plateau de jeu \reference{etat_jeu}.

La sélection des actions est effectuée dans une nouvelle fenêtre \reference{actions}, dans laquelle
le joueur va pouvoir sélectionner les 6 actions qu'il veut effectuer, ainsi que
le choix de la région s'il choisit de poser une délégation.

À la fin de la partie, une fenêtre s'ouvre et affiche les scores de chaque joueur
dans chacun des domaines (Politique, Religieux et Économique), ainsi que le vainqueur \reference{resultats}.

\subsubsection{Création des intelligences artificielles}

Nous avons tout d'abord réalisé une IA aléatoire, dans un but de rapidité et de test.
En effet, celle-ci nous a permis de réaliser une partie, et de vérifier que les différentes actions 
s'effectuent correctement. Elle choisit 6 actions de manière aléatoire.
Cette IA a été utile pour tester l'affichage de l'interface graphique, et sa
mise à jour durant l'exécution des tours : les actions sont effectuées une
à une, avec une courte pause entre chaque, pour permettre de bien suivre le déroulement du jeu.

Nous avons ensuite commencé le développement de l'IA basée sur un algorithme évolutionnaire.
Cette dernière utilise l'algorithme évolutionnaire pour réaliser deux choses différentes:
\begin{itemize}
	\item Choisir sa position de départ.
	\item Planifier les actions qu'elle va réaliser durant le tour.
\end{itemize}

L'algorithme va réaliser, pour chaque génération, une suite d'actions:
\begin{enumerate}
	\item Une sélection: on sélectionne les individus qui vont participer à la reproduction.
	\item Des mutations de la population des enfants: mutation ou crossing-over, suivant un ratio donné.
	\item Évaluer la population des enfants, en calculant leur fitness.
	\item effectuer un remplacement: on garde les meilleurs individus pour composer la nouvelle population de parents.
\end{enumerate}

\lstinputlisting[language=Java, frame=tb, caption= {Déroulement de l'algorithme évolutionnaire}]{code/algo_evol.java}
